%%%%%%%%%%%%%%%%%%%%%%%%%%%%%%%%%%%%%%%%%
% University Assignment Title Page 
% LaTeX Template
% Version 1.0 (27/12/12)
%
% This template has been downloaded from:
% http://www.LaTeXTemplates.com
%
% Original author:
% WikiBooks (http://en.wikibooks.org/wiki/LaTeX/Title_Creation)
%
% License:
% CC BY-NC-SA 3.0 (http://creativecommons.org/licenses/by-nc-sa/3.0/)
% 
% Instructions for using this template:
% This title page is capable of being compiled as is. This is not useful for 
% including it in another document. To do this, you have two options: 
%
% 1) Copy/paste everything between \begin{document} and \end{document} 
% starting at \begin{titlepage} and paste this into another LaTeX file where you 
% want your title page.
% OR
% 2) Remove everything outside the \begin{titlepage} and \end{titlepage} and 
% move this file to the same directory as the LaTeX file you wish to add it to. 
% Then add \input{./title_page_1.tex} to your LaTeX file where you want your
% title page.
%
%%%%%%%%%%%%%%%%%%%%%%%%%%%%%%%%%%%%%%%%%

%----------------------------------------------------------------------------------------
%	PACKAGES AND OTHER DOCUMENT CONFIGURATIONS
%----------------------------------------------------------------------------------------

\documentclass[11pt]{article}
\usepackage[margin=1.0in]{geometry}
\usepackage{graphicx}
\usepackage{caption}
\usepackage{subcaption}
\usepackage{float}
\DeclareGraphicsExtensions{.png}
\begin{document}

\begin{titlepage}

\newcommand{\HRule}{\rule{\linewidth}{0.5mm}} % Defines a new command for the horizontal lines, change thickness here

\center % Center everything on the page
 
%----------------------------------------------------------------------------------------
%	HEADING SECTIONS
%----------------------------------------------------------------------------------------

\textsc{\LARGE Wright State University}\\[1.5cm] % Name of your university/college
\textsc{\Large CS 7900}\\[0.5cm] % Major heading such as course name
\textsc{\large Proposal}\\[0.5cm] % Minor heading such as course title

%----------------------------------------------------------------------------------------
%	TITLE SECTION
%----------------------------------------------------------------------------------------

\HRule \\[0.4cm]
{ \huge \bfseries A Sentiment Analysis Application for Tweets}\\[0.4cm] % Title of your document
\HRule \\[1.5cm]
 
%----------------------------------------------------------------------------------------
%	AUTHOR SECTION
%----------------------------------------------------------------------------------------

\begin{minipage}{0.4\textwidth}
\begin{flushleft} \large
\emph{Author:}\\
Daniel \textsc{McIntyre} U00653482 % Your name

Nathan \textsc{Rude} U00637118% Your name

\end{flushleft}
\end{minipage}
~
\begin{minipage}{0.4\textwidth}
\begin{flushright} \large
\emph{Supervisor:} \\
Dr. Tanvi \textsc{Banerjee} % Supervisor's Name
\end{flushright}
\end{minipage}\\[4cm]

% If you don't want a supervisor, uncomment the two lines below and remove the section above
%\Large \emph{Author:}\\
%John \textsc{Smith}\\[3cm] % Your name

%----------------------------------------------------------------------------------------
%	DATE SECTION
%----------------------------------------------------------------------------------------

{\large \today}\\[3cm] % Date, change the \today to a set date if you want to be precise

%----------------------------------------------------------------------------------------
%	LOGO SECTION
%----------------------------------------------------------------------------------------

%\includegraphics{Logo}\\[1cm] % Include a department/university logo - this will require the graphicx package
 
%----------------------------------------------------------------------------------------

\vfill % Fill the rest of the page with whitespace

\end{titlepage}
%--------------------------------------
%  Content Sections
%------------------------------------

\section{Goal}
Social media is a very ingrained component of our culture. In this day and age, there are social media users all over the world. In fact, social data is a very rich source of data in today’s information age. For this project, we aim to leverage the power of social data from Twitter. As a micro-blogging outlet, users regularly churn out bite-sized chunks of information that can be analysed for information on just about any topic. By aggregating tweets about a certain topic from Twitter, one can get a general sentiment from the populous about that certain topic. 

Since tweets can be retrieved in real time, the sentiment of current and trending topics, such as presidential debates and debate topics, can be evaluated to determine how the populous feels about each topic. Understanding and utilizing this user feedback can lead towards policy change.

\section{Objective}
We plan to create a web application that allows users to view the sentiment of topics on Twitter. Through the use of two API’s, we will 1) retrieve tweets from Twitter and 2) perform sentiment analysis on these tweets to aggregate the general sentiment about a topic. 

\section{Approach}
This application can be split into two parts: a frontend for the user to select topics and view the sentiment, and a backend for the text retrieval and sentiment analysis to take place. 

The user interface will be developed with HTML, CSS and Javascript, technologies that have been learned in class. Using these technologies we can create a dynamic layout that, given a search topic, can display several retrieved tweets with their individual sentiment, as well as the overall sentiment of a topic.

The backend will handle the routing and logic for extracting and analysing tweets. Backend technologies will be used based on compatibility with the chosen APIs. For simplicity, sentiment analysis will be performed at tweet level, rather than at phrase or entity level. 

\section{Evaluation}
To evaluate the sentiment analysis component of the application, a selection of labeled tweets will be used and several metrics, such as precision and recall, will be computed to establish the performance of our algorithm.  

\section*{\centering{Milestones \& Timeline}}
\begin{center}
  \begin{tabular}{|p{3.5cm}|p{6cm}|}
    \hline 
    \textbf{Week of} & \textbf{Planned Tasks} \\ \hline \hline
    February 29th, 2016 & Write Project Proposal \\ \hline
    March 7th, 2016 & Project Proposal Presentation \newline
						Start Constructing Project Report \newline
						Research APIs \\ \hline
    March 14th, 2016 & Continued API Research \newline
						Design Application Architecture \\ \hline
    March 21st, 2016 & Setup Backend \newline
						Design UI \\ \hline
    March 28th, 2016 & Develop Backend \newline
						Integrate APIs \\ \hline
    April 4th, 2016 & Integrate APIs \newline
						Develop Front End \\ \hline
    April 11th, 2016 & Clean Up Front End \\ \hline
    April 18th, 2016 & Final Integration \newline
    					Testing and Evaluation \newline
						Finish Project Report \\ \hline
    April 25th, 2016 & Finalize Project Report \newline
						Finalize Demo \newline
						Final Demo \\ \hline
  \end{tabular}
\end{center}

\pagebreak

\bibliography{egbib}
\bibliographystyle{plain}

\end{document}